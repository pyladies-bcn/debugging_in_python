%\documentclass{beamer}
%
%\title{Introduction to debugging in Python}
%\author{PyladiesBCN - @lpmayos}
%\date{February 26, 2015}
%
%\usetheme{pyladies}
%
%\begin{document}
%\maketitle


\documentclass{beamer}
\usetheme{pyladies}

\usepackage[utf8]{inputenc}
\usepackage[T1]{fontenc}

%% Use any fonts you like.
\usepackage{helvet}

% code highlighting
\usepackage{listings}
\usepackage{color}
 
\definecolor{codegreen}{rgb}{0,0.6,0}
\definecolor{codegray}{rgb}{0.5,0.5,0.5}
\definecolor{codepurple}{rgb}{0.58,0,0.82}
\definecolor{backcolour}{rgb}{0.95,0.95,0.92}
 
\lstdefinestyle{mystyle}{
    backgroundcolor=\color{backcolour},   
    commentstyle=\color{codegreen},
    keywordstyle=\color{magenta},
    numberstyle=\tiny\color{codegray},
    stringstyle=\color{codepurple},
    basicstyle=\scriptsize,
    breakatwhitespace=false,         
    breaklines=true,                 
    captionpos=b,                    
    keepspaces=true,                 
    numbers=left,                    
    numbersep=5pt,                  
    showspaces=false,                
    showstringspaces=false,
    showtabs=false,                  
    tabsize=2
}
 
\lstset{style=mystyle}




\title{Introduction to debugging in Python}
\subtitle{(+ CodeFights!)}
\author{@lpmayos}
\date{\today}
\institute{PyladiesBCN}

\begin{document}

\begin{frame}[plain,t]
\titlepage
\end{frame}

\section{Why should we learn to debug?}

\begin{frame}
\frametitle{Why should we learn to debug?}	

\begin{quote}
Is this code snippet similar to your tried-and-true debugging techniques? Yeah, that used to be me too.
\end{quote}

\lstinputlisting[language=Python]{code/caos.py}


{\normalsize Yeah, the print command works, but... What if you don't know where to look? What if you are facing a huge piece of code that was written by a drunken elephant riding a tricyce? There is a better way!}

\end{frame}






\section{Python debugging tools}

\subsection{Logging}
\begin{frame}
\frametitle{Logging}	
\begin{quote}
	{\Large If you do ever litter your code with print statements stop now. Use logging.debug instead. You'll be able to reuse that later, disable it altogether and so on ...}
\end{quote}
Take a look at the logging module \url{https://docs.python.org/2/library/logging.html}
\end{frame}

\subsection{Tracing}
\begin{frame}
\frametitle{Tracing}	
\begin{quote}
The trace module allows you to trace program execution, generate annotated statement coverage listings, print caller/callee relationships and list functions executed during a program run. It can be used in another program or from the command line.
\end{quote}
Take a look at the trace module \url{https://docs.python.org/2/library/trace.html}\\
i.e.\\
\lstinputlisting[language=bash]{code/tracing.txt}
\end{frame}

\subsection{Debugging}
\begin{frame}
\frametitle{Debugging}	
\begin{itemize}
 \item  \textbf{pdb module}: defines an interactive source code debugger for Python programs (\url{https://docs.python.org/2/library/pdb.html})
 \item \textbf{ipdb module}: exports functions to access the IPython debugger, which features tab completion, syntax highlighting, better tracebacks, better introspection with the same interface as the pdb module (\url{https://pypi.python.org/pypi/ipdb})
 \end{itemize} 
\end{frame}





\section{PDB and IPDB}

\subsection{ipdb.set\_trace()}
\begin{frame}
\frametitle{Getting started with \textit{\textbf{ipdb.set\_trace()}}}
\begin{table}
    \begin{tabular}{p{4.7cm}p{4.7cm}}
        \lstinputlisting[language=Python]{code/max_of_three.py}&
        \lstset{basicstyle=\ttfamily\tiny}
        \lstinputlisting[language=bash]{code/max_of_three_console.txt}
    \end{tabular} 
\end{table}
\end{frame}

\subsection{autocomplete}
\begin{frame}
\frametitle{Take the advantages offered by ipdb with \textit{\textbf{TAB}} (autocomplete)}
\begin{table}
    \begin{tabular}{p{4.7cm}p{4.7cm}}
        \lstset{basicstyle=\ttfamily\tiny}
        \lstinputlisting[language=Python]{code/divisible_3_4.py}&
        \lstset{basicstyle=\ttfamily\tiny}
        \lstinputlisting[language=bash]{code/divisible_3_4_console.txt}
    \end{tabular} 
\end{table}
\end{frame}

\subsection{next}
\begin{frame}
\frametitle{Execute the next statement with \textit{\textbf{n}} (next)}
\begin{table}
    \begin{tabular}{p{4.7cm}p{4.7cm}}
        \lstinputlisting[language=Python]{code/max_of_three.py}&
        \lstset{basicstyle=\ttfamily\tiny}
        \lstinputlisting[language=bash]{code/max_of_three_console2.txt}
    \end{tabular} 
\end{table}
\end{frame}

\subsection{repeat}
\begin{frame}
\frametitle{Repeat the last debugging command with \textit{\textbf{ENTER}}}
\begin{table}
    \begin{tabular}{p{4.7cm}p{4.7cm}}
        \lstinputlisting[language=Python]{code/max_of_three.py}&
        \lstset{basicstyle=\ttfamily\tiny}
        \lstinputlisting[language=bash]{code/max_of_three_console3.txt}
    \end{tabular} 
\end{table}
\end{frame}

\subsection{quit}
\begin{frame}
\frametitle{Help! How do I quit? with \textit{\textbf{q}} (quit)}
\begin{table}
    \begin{tabular}{p{4.7cm}p{4.7cm}}
        \lstinputlisting[language=Python]{code/max_of_three.py}&
        \lstset{basicstyle=\ttfamily\tiny}
        \lstinputlisting[language=bash]{code/max_of_three_console4.txt}
    \end{tabular} 
\end{table}
\end{frame}

\subsection{print}
\begin{frame}
\frametitle{Print the value of a variable with \textit{\textbf{p}} (print)}
\begin{table}
    \begin{tabular}{p{4.7cm}p{4.7cm}}
        \lstinputlisting[language=Python]{code/max_of_three.py}&
        \lstset{basicstyle=\ttfamily\tiny}
        \lstinputlisting[language=bash]{code/max_of_three_console5.txt}
    \end{tabular} 
\end{table}
\end{frame}

\subsection{continue}
\begin{frame}
\frametitle{Turning off the (Pdb) prompt with \textit{\textbf{c}} (continue)}
\begin{table}
    \begin{tabular}{p{4.7cm}p{4.7cm}}
        \lstinputlisting[language=Python]{code/max_of_three.py}&
        \lstset{basicstyle=\ttfamily\tiny}
        \lstinputlisting[language=bash]{code/max_of_three_console6.txt}
    \end{tabular} 
\end{table}
\end{frame}

\subsection{list}
\begin{frame}
\frametitle{See where you are with \textit{\textbf{l}} (list)}
\begin{table}
    \begin{tabular}{p{4.7cm}p{4.7cm}}
        \lstinputlisting[language=Python]{code/max_of_three.py}&
        \lstset{basicstyle=\ttfamily\tiny}
        \lstinputlisting[language=bash]{code/max_of_three_console7.txt}
    \end{tabular} 
\end{table}
\end{frame}

\subsection{step into}
\begin{frame}
\frametitle{Step into subroutines with \textit{\textbf{s}} (step into)}
\begin{table}
    \begin{tabular}{p{4.7cm}p{4.7cm}}
        \lstinputlisting[language=Python]{code/subroutines.py}&
        \lstset{basicstyle=\ttfamily\tiny}
        \lstinputlisting[language=bash]{code/subroutines_console.txt}
    \end{tabular} 
\end{table}
\end{frame}

\subsection{return}
\begin{frame}
\frametitle{Continue to the end of the current subroutine with \textit{\textbf{r}} (return)}
\begin{table}
    \begin{tabular}{p{4.7cm}p{4.7cm}}
        \lstinputlisting[language=Python]{code/subroutines.py}&
        \lstset{basicstyle=\ttfamily\tiny}
        \lstinputlisting[language=bash]{code/subroutines_console2.txt}
    \end{tabular} 
\end{table}
\end{frame}

\subsection{some advice}
\begin{frame}
\frametitle{...just be a little careful!}
\begin{table}
    \begin{tabular}{p{4.7cm}p{4.7cm}}
        \lstinputlisting[language=Python]{code/max_of_three.py}&
        \lstset{basicstyle=\ttfamily\tiny}
        \lstinputlisting[language=bash]{code/max_of_three_console8.txt}
        What happens is that pdb attempts to execute the "b" command and it interprets the rest of the line as an argument to the "b" command
    \end{tabular} 
\end{table}\end{frame}






\section{Sources}
\begin{frame}
\frametitle{Sources}	
\begin{itemize}
    \item Debugging in Python, 2009 post by Steve Ferg: \url{https://pythonconquerstheuniverse.wordpress.com/2009/09/10/debugging-in-python/}
    \item Python debugging tools, 2013 post by ionel's codelog: \url{http://blog.ionelmc.ro/2013/06/05/python-debugging-tools/}
    \item Debugging Python Like a Boss  \url{https://zapier.com/engineering/debugging-python-boss/}
\end{itemize}
\end{frame}





\section{Let's practice!}
\begin{frame}
\frametitle{Let's practice!}	

\begin{enumerate}
    \item First solve the \textbf{exercises} I prepared for this session (based on some CodeFights problems)
        \begin{itemize}
            \item Please download them from \url{https://github.com/pyladies-bcn/debugging_in_python}
        \end{itemize}
    \item Then let's play \textbf{CodeFights}
        \begin{itemize}
            \item If you fail a problem, try to copy the code in a file and use \textbf{ipdb} to solve it!
            \item or we can play a \textbf{Tournament} instead!
        \end{itemize}
\end{enumerate}

\end{frame}



\ThankYouFrame

\end{document}